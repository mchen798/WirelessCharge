\documentclass[a4paper]{article}
\usepackage{amsmath}
\usepackage{amssymb}
\title{Blockchain Based Energy Trade for Shared Wireless Chargers}
\author{Junbo Wang, Chen Qiu, Hongzhi Xiao}
\begin{document}
\maketitle
\section*{abstract}
\section{Introduction}
Wireless Charging Technology is getting popular\\
 Power consuming in smart phone is huge and battery is not good (cannot afford for one day usage)\\
 Bring an additional battery can be a burden to users\\
 Shared wireless charging becomes a new trend
 
\section{Problem Define}
It is hard to record the energy trade transactions since people is moving, and use services in different areas. \\
Centralized way lose its function 
\subsection{Process of Blockchain Network}
The building should include the aggregator. aggregator as a node of the blockchain network.\\
In this layer, each aggregator could validate transactions and blocks, also they could generate transactions and blocks. \\
They may have problems:\\
\begin{itemize}
	\item The aggregator may very expensive.
	\item The benefit not enough for the cost.
\end{itemize}

\subsection{Cloud Services}
Cloud is provide computing and technical support.
In this modeling, shop could got cloud computation resource if they don’t want set aggregator inside. But shop need have a sever to storage the blockchain and some data.\\
The price of the cloud could changed. It depend on the shop’s benefit. Another satisfaction problem between cloud and shop.\\
The cloud could be a center of the modeling, because the block size could be changed by cloud macro-control. \\
They may have problems:
\begin{itemize}
	\item The cloud should save the blockchain or not?
	\item If need macro-control, all the node should connect with cloud. Looks like the could will be centralization.
\end{itemize}
\subsection{User Satisfaction}
\begin{itemize}
	\item Assuming the $S^{t}  \geq S_0$	($S^t$ is user satisfaction, $S_0$ is minimum user satisfaction).
	\item $E_{ik}$ is actual power consumption. $E'$ is user demand power consumption.
	\item $E_{ik} = e*t$.
	\item $e$ means charging efficiency, $t$ means charging time.
	\item $S^t =min(E_{ik}/E',1)$.
	\item In this equation, user could get percentage of demand power consumption, if user get full $E'$, the satisfaction became $100\%$.
	
\end{itemize}

\subsection{Aggregator Profit}
The aggregator need maximum profit. $\sum_{u\subset k}^{}a_{ki}$ means how many users connect with the aggregator.
\begin{eqnarray}
	P_t &=& \sum_{u\subset k}^{}a_{ki}E_{ik}( \frac{c_0}{e}-CE)-C_{POW} \label{ap} \\
	a_{ki} &=& \left(\begin{array}{cc}
	0  k\nleftarrow i \\ 1  k\leftarrow i \end{array}\right)
\end{eqnarray}
$C_0$ is charging price per second, $C_E$ means cost of energy per $kW\cdot h$. We suppose 
\begin{equation}
	C_{POW}=r_i \cdot t
\end{equation}
From the Paper $"POS  VS POW"$, we found the following equations to calculate the cost of $POW (C_{POW})$ .\\
Target difficulty is $D\subset [1,M]$, the time of mining express by $T_r$. \\
One miner case, possibility of find a block: $P \{T_r\leq t\}=1-exp(-\frac{rt}{D})$.\\
Assuming $n$ miners, each miner have $r$ operation per second. 
$$P \{T=min(T_1,T_2,T_3,T_n\leq T) \}=1-exp(-\frac{t}{D}\sum_{i=1}^{n}r_i )$$
In this part, we got variable $r_i$, $t$ and $a_{ki}$.

\section{Solutions}
There are two equations. 
For the aggregator, it should consider maximum profit.
$$Max P(t)$$
For the user, we assume the satisfaction should not less than $S_0$  
$$S_t\geq S_0$$
We try to find a group of solutions to get optimal two equations.
We got 
\begin{itemize}
	\item $t \leq threshold1$
	\item $P_t\geq100\%$
	\item $Tradeoff_{ri}\leq treshold2$
\end{itemize}


\section{Discussion}
\section{Conclusion}
\section{Future Work}
\end{document}